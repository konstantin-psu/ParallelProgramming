\documentclass{article}

\usepackage{fancyhdr}
\usepackage{extramarks}
\usepackage{amsmath}
\usepackage{amsthm}
\usepackage{amsfonts}
\usepackage{tikz}
\usepackage[plain]{algorithm}
\usepackage{algpseudocode}
\usepackage{enumerate}

\usepackage{listings}
\usepackage{xcolor}
\usepackage{forest}
\usepackage[shortlabels]{enumitem}
     \setlist[enumerate, 1]{1\textsuperscript{o}}
\lstset { %
    language=C++,
    backgroundcolor=\color{black!5}, % set backgroundcolor
    basicstyle=\footnotesize,% basic font setting
}

%\usetikzlibrary{automata,positioning}
\usetikzlibrary{positioning,shapes,shadows,arrows,automata}

%
% Basic Document Settings
%

\topmargin=-0.45in
\evensidemargin=0in
\oddsidemargin=0in
\textwidth=6.5in
\textheight=9.0in
\headsep=0.25in

\linespread{1.1}

\pagestyle{fancy}
\lhead{\hmwkAuthorName}
\rhead{ (\hmwkClassInstructor\ \hmwkClassTime): \hmwkTitle}
\lfoot{\lastxmark}
\cfoot{\thepage}

\renewcommand\headrulewidth{0.4pt}
\renewcommand\footrulewidth{0.4pt}

\setlength\parindent{0pt}

%
% Create Problem Sections
%

\newcommand{\enterProblemHeader}[1]{
    \nobreak\extramarks{}{Problem \arabic{#1} continued on next page\ldots}\nobreak{}
    \nobreak\extramarks{Problem \arabic{#1} (continued)}{Problem \arabic{#1} continued on next page\ldots}\nobreak{}
}

\newcommand{\exitProblemHeader}[1]{
    \nobreak\extramarks{Problem \arabic{#1} (continued)}{Problem \arabic{#1} continued on next page\ldots}\nobreak{}
    \stepcounter{#1}
    \nobreak\extramarks{Problem \arabic{#1}}{}\nobreak{}
}

\setcounter{secnumdepth}{0}
\newcounter{partCounter}


\newcommand{\hmwkTitle}{Assignment \#1}
\newcommand{\hmwkDueDate}{April 20, 2016}
\newcommand{\hmwkClass}{CS515 Parallel Programming}
\newcommand{\hmwkClassTime}{Spring 2016}
\newcommand{\hmwkClassInstructor}{Jingke Li}
\newcommand{\hmwkAuthorName}{Konstantin Macarenco}


\title{
    \vspace{2in}
    \textmd{\textbf{\hmwkClass:\ \hmwkTitle}}\\
    \normalsize\vspace{0.1in}\small{Due\ on\ \hmwkDueDate\ at 11:59pm}\\
    \vspace{0.1in}\large{\textit{\hmwkClassInstructor\ \hmwkClassTime}}
    \vspace{3in}
}

\author{\textbf{\hmwkAuthorName}}
\date{}

\renewcommand{\part}[1]{\textbf{\large Part \Alph{partCounter}}\stepcounter{partCounter}\\}

%
% Various Helper Commands
%

% Useful for algorithms
\newcommand{\alg}[1]{\textsc{\bfseries \footnotesize #1}}

% For derivatives
\newcommand{\deriv}[1]{\frac{\mathrm{d}}{\mathrm{d}x} (#1)}

% For partial derivatives
\newcommand{\pderiv}[2]{\frac{\partial}{\partial #1} (#2)}

% Integral dx
\newcommand{\dx}{\mathrm{d}x}

% Alias for the Solution section header
\newcommand{\solution}{\textbf{\large Solution}}

% Probability commands: Expectation, Variance, Covariance, Bias
\newcommand{\E}{\mathrm{E}}
\newcommand{\Var}{\mathrm{Var}}
\newcommand{\Cov}{\mathrm{Cov}}
\newcommand{\Bias}{\mathrm{Bias}}

\begin{document}

\maketitle

\pagebreak

% Write a short (one-page) summary covering your experience with this assignment: what issues you
% encoun- tered, how you resolved them, and what lessons you’ve learned. Make a zip file containing your
% program and your write-up, and submit it through the Dropbox on the D2L site.
%\begin{enumerate}[(a), leftmargin = 0.7cm, nosep]
\textbf{Quick sort summary}\\
The main complication of Quick Sort over Producer Consumer is that Producer consumer has only
one Producer and many Consumers, whereas this Quick Sort implementation requires many Producers
to many Consumers.\\

\textbf{Termination condition}: 

Task queue being empty as termination condition, this is the most obvious
solution, however it has a major drawback (as mentioned in the assignment):

Race condition with the quicksort method (Producer). It is possible to have a situation
when queue emptiness is checked when quicksort is started but not yet added any tasks in
the queue, in this case workers exit prematurely.

Another termination condition is needed. We have two options, 
\begin{enumerate}[1.]
\item Global variable to count producers
\item Special task that indicates that no more tasks will be produced, for example a task with negatives
values in place high and low. 
\end{enumerate}
Later won't work since qucksort, unlike simple producer-consumer, builds binary tree, which size will vary depending 
on the size of the array. Each node will create ``final'' task, that only
indicates that this particular node is finished but not necessarily the entire sort, and each
node has no knowledge about the tree size.\\

On the other hand, a global variable (numProducers) that holds number of active quicksorts solves this problem. Each time before
quicksort call the variable increased by 1, and decreased when branch dies, so we always know
if sort is over $numProducers + queueIsEmpty$ is enough for workers to terminate correctly. 
There is one drawback in using shared
variable, it creates additional bottleneck, since we have M producers updating it, and N workers
reading - every update and read must be lock protected. This is not as noticeable, since the
queue itself is a bottleneck.
\\

\textbf{Blocking vs non Blocking queue}

Another important decision is whether to make workers sleep, or spin wait if there is no tasks available.
This problem doesn't apply to producers, since we make an assumption that queue has ``unlimited'' size.

Sleep has one potential pitfall - last qsort is still in session but it already queued it's task,
which was consumed before qsort returned, hence this will put all threads that are in the
middle of this operation into infinite sleep. Another disadvantage is that since qsort never
sleeps there is almost no wait on the task queue, performing context switch on a worker is more
expansive than spin wait in this case.\\

\textbf{Other Problems}\\
I had hard time synchronizing threads, since at least one termination condition must be
unsatisfied at the start.  I ended up setting $numProducers$ to 1 in main() so all workers will
spin wait while queue is empty.\\

Another problem is debugging, it is becoming extremely to hard as number of threads is
increasing.
%CCCCOOOO

%CCCCOOO1
\pagebreak
\textbf{Producer consumer}\\

In this problem I used standard producer-consumer solution with blocking queue, with a special
task as a termination condition, that producer adds to the queue as soon as all task are
produced. In this case it is a task with different high and low values.
As soon as the task is in the queue producer signals one of the consumers and quits. 

Each consumer will get the last task, add it back to the queue, and signal the next consumer.

One advantage of using the queue over a global variable as a termination condition is that it
doesn't add any additional penalties since each consumer already locks the queue to get next task.

\end{document}
